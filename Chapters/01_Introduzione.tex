\chapter{Cinematica del punto}

\section{Introduzione}
La meccanica riguarda lo studio del moto di un corpo: spiega la relazione tra le cause che lo generano e le sue caratteristiche, esprimendola con leggi quantitative.

	\subsection{Punto materiale}
	Un punto materiale o particella \`e un corpo privo di dimensioni: le sue dimensioni sono trascurabili rispetto a quelle dello spazio in cui pu\`o muoversi o degli altri corpi con cui pu\`o interagire.

	\subsection{Movimenti di un corpo esteso}
	\begin{multicols}{2}
		\begin{itemize}
			\item Traslazione: il corpo esteso si muove come un punto materiale.
			\item Rotazioni.
			\item Vibrazioni.
		\end{itemize}
	\end{multicols}

	\subsection{Cinematica}
	Si intende per cinematica una parte della meccanica che studia il moto senza considerare le forze che entrano in gioco.

	\subsection{Determinare il moto}
	Il moto di un punto materiale \`e determinato se \`e nota la sua posizione in funzione del tempo in un determinato sistema di riferimento.

		\subsubsection{Sistema di riferimento cartesiano}
		In un sistema di riferimento cartesiano il la posizione di un corpo \`e data dalle sue coordinate $x(t)$, $y(t)$, $z(t)$, espresse in funzione del tempo.
		Altri sistemi di riferimento fanno uso delle coordinate polari.

	\subsection{Traiettoria}
	La traiettoria \`e il luogo dei punti occupati successivamente dal punto in movimento.
	Costituisce una curva continua nello spazio.

	\subsection{Grandezze fondamentali}
	Nella cinematica le grandezze fondamentali sono:
	\begin{multicols}{2}
		\begin{itemize}
			\item Spazio.
			\item Velocit\`a.
			\item Accelerazione.
			\item Tempo o la variabile indipendente.
		\end{itemize}
	\end{multicols}

	\subsection{Quiete}
	La quiete \`e un tipo di moto in cui le coordinate rimangono costanti, pertanto velocit\`a ed accelerazione sono nulle.

\section{Moto rettilineo}

	\subsection{Descrizione}
	Il moto rettilineo si svolge lungo una retta su cui vengono fissati arbitrariamente un'origine e un verso.
	Il moto del punto pu\`o essere descrivibile tramite una coordintata $x(t)$.

	\subsection{Rappresentazione}
	Le misure ottenute da un'osservazione di un moto rettilineo per tempo e spazio possono essere rappresentate in un sistema a due assi cartesiani: sulle ordinate i valori di $x$ e su quello delle ascisse il tempo $t$ corrispondente.
	Questo viene detto diagramma orario.

	\subsection{Velocit\`a}

		\subsubsection{Velocit\`a media}
		Se al tempo $t=t_1$ il punto si trova nella posizione $x=x_1$ e al tempo $t=t_2$ nella posizione $x=x_2$, $\Delta x = x_2 - x_1$ rappresenta lo spazio percorso nell'intervallo di tempo $\Delta t = t_2 - t_1$.
		La rapidit\`a con cui avviene lo spostamento viene caratterizzata dalla velocit\`a media:
		$$v_m = \dfrac{\Delta x}{\Delta t} = \dfrac{x_2 - x_1}{t_2 - t_1}$$

		\subsubsection{Velocit\`a istantanea}
		Per ricavare informazioni riguardo le caratteristiche del moto si pu\`o suddividere $\Delta x$ in numerosi piccoli intervalli $(\Delta x)_i$ percorsi in altrettanti piccoli intervalli di $\Delta t$ $(\Delta t)_i$.
		Si nota come le corrispondenti velocit\`a medie sono $v_i = \frac{(\Delta x)_i}{(\Delta t)_i}$, diverse tra di loro e da $v_m$.
		Questo avviene in quanto in un generico moto rettilineo la velocit\`a non \`e costante nel tempo.
		Suddividendo $\Delta x$ in un numero elevatissimo di intervallini $dx$ percorsi nel tempo $dt$ si pu\`o definire la velocit\`a istantanea ad un istante $t$ del punto in movimento come:
		$$v = \lim\limits_{\Delta t \rightarrow 0} \dfrac{\Delta x}{\Delta t} = \dfrac{dx}{dt}$$
		La velocit\`a istantanea rappresenta pertanto la rapidit\`a di variazione temporale della posizione nell'istante $t$ considerato.
		Il segno indica il verso del moto sull'asse.
		Pu\`o inoltre essere espressa come funzione del tempo $v(t)$.

			\paragraph{Moto rettilineo uniforme}
			Si intende per moto rettilineo uniforme un tipo di moto rettilineo in cui la velocit\`a \`e costante.

			\paragraph{Ottenere la velocit\`a}
			Nota la legge oraria $x(t)$ si pu\`o otenere la velocit\`a istantanea con l'operazione di derivazione.

			\paragraph{Ottenere la legge oraria}
			Nota la dipendenza del tempo della velocit\`a istantanea $v(t)$ si pu\`o ottenere la legge oraria $x(t)$.
			Supponendo che il punto si trovi in $x$ al tempo $t$ e nella posizione $x+dx$ in $t+dt$ da $v=\frac{dx}{dt}$ si nota come lo spostamento infinitesimo $dx$ \`e uguale al prodotto del tempo $dt$ impiegato a percorrerlo per il valore della velocit\`a al tempo $t:dx=v(t)dt$, qualunque sia la dipendenza della velocit\`a dal tempo.
			Lo spostamento complessivo sulla retta su cui si muove il punto in un intervallo finito $\Delta t = t - t_0$ \`e dato dalla somma di tutti i successivi valori $dx$.
			Si utilizza l'operazione di integrazione:
			\begin{align*}
				&\Delta x = \int_{x_0}^x dx = \int_{t_0}^t v(t)dt\\
				&x - x_0 = \int_{t_0}^t v(t)dt\\
				&x = x_0 +\int_{t_0}^tv(t)dt
			\end{align*}
			Si ottiene pertanto la relazione generale che permette il calcolo dello spazio percorso nel moto rettilineo:
			$$x(t) = x_0 + \int_{t_0}^t v(t)dt$$
			Dove $x_0$ rappresenta la posizione iniziale del punto occupata nell'istante $t_0$
			Si noti come $\Delta x$ rappresenta la somma algebrica degli spostamenti.

		\subsubsection{Relazione tra velocit\`a media e istantanea}
		Ricordando che $v_m = \frac{x-x_0}{t-t_0}$, la relazione tra velocit\`a media e istantanea:
		$$v_m = \dfrac{1}{t-t_0}\int_{t_0}^{t}v(t)dt$$

		\subsubsection{Legge oraria del moto rettilineo uniforme}
		Considerando il moto rettilineo uniforme in cui $v$ \`e costante si ha:
		\begin{align*}
			x(t) &= x_0 + v\int_{t_0}^t dt\\
			     &=x_0 + v(t-t_0)\\
			     &=x_0 + vt\qquad\qquad se\ t_0 = 0
		\end{align*}
		Si nota pertanto come nel moto rettilineo uniforme lo spazio \`e una funzione lineare del tempo e la velocit\`a istantanea coincide con la velocit\`a media.

	\subsection{Accelerazione}
	La velocit\`a $v(t)$ varia in un determinato $\Delta t$ di una quantit\`a $\delta v$.

		\subsubsection{Accelerazione media}
		Analogamente alla velocit\`a media si definisce l'accelerazione media come
		$$a_m = \dfrac{\Delta v}{\Delta t}$$

		\subsubsection{Accelerazione istantanea}
		Si definisce accelerazione istantanea come la rapidit\`a di variazione temporale della celocit\`a come:
		$$a = \dfrac{dv}{dt} = \dfrac{d^2 x}{dt^2}$$

		\subsubsection{Significato fisico dell'accelerazione}
		\begin{multicols}{2}
			\begin{itemize}
				\item $a=0$: velocit\`a costante, moto rettilineo uniforme.
				\item $a>0$: la velocit\`a cresce nel tempo.
				\item $a<0$: la velocit\`a decresce nel tempo.
			\end{itemize}
		\end{multicols}

		\subsubsection{Ottenere la velocit\`a}
		Data una $a(t)$ si ricava $v(t)$:
		\begin{align*}
			dv &=a(t)dt\\
			\Delta v &= \int_{v_0}^v dv\\
			   &=\int_{t_0}^t a(t)dt\\
		\end{align*}
		Pertanto:
		$$v(t) = v_0 + \int_{t_0}^t a(t)dt$$

		\subsubsection{Moto rettilineo uniformemente accelerato}
		Si intende per moto rettilineo uniformemente accelerato un moto in cui l'accelerazione \`e costante durante il moto.

			\paragraph{Dipendenza della velocit\`a dal tempo}
			La dipendenza della velocit\`a dal tempo \`e lineare:
			\begin{align*}
				v(t) &=v_0+a(t-t_0)\\
				v(t) &=v_0+at\qquad\qquad se\ t_0 = 0
			\end{align*}

			\paragraph{Dipendenza della posizione dal tempo}
			Lo spazio \`e una funzione quadratica del tempo:
			\begin{align*}
				x(t) &= x_0 +\int_{t_0}^t [v_0 + a(t-t_0)]dt\\
				     &= x_0 + \int_{t_0}^t v_0dt + \int_{t_0}^t a(t-t_0)dt\\
				x(t) &= x_0 + v(t-t_0) +\dfrac{1}{2}a(t-t_0)^2\\
				     &= x_0 + v_0t +\dfrac{1}{2}at^2\qquad\qquad se\ t_0 = 0
			\end{align*}

			\paragraph{Dipendenza dell'accelerazione dalla posizione}
			Nota la dipendenza dell'accelerazione dalla posizione, ovvero $a(x)$ si pu\`o ricavare il valore della velocit\`a in ogni posizione $x$ o $v(x)$.
			Questo avviene considerando le funzioni di funzione.
			Se ad un istante $t$ il punto occupa una posizione $x$ con velocit\`a $v$ e accelerazione $a$ si possono pensare come funzioni della posizione e
			$$v(t) = v[x(t)]$$
			$$a(t) = a[x(t)]$$
			Derivando la prima rispetto al tempo e sfruttando la regola di derivazione delle funzioni di funzioni:
			\begin{align*}
				a[x(t)] &= \dfrac{d}{dt}v[x(t)]\\
				        &= \dfrac{d}{dt}\dfrac{d}{dt}\dfrac{dv}{dx}\dfrac{dx}{dt}\\
					&= \dfrac{dv}{dx}\dfrac{dx}{dt}\\
				a  	&= \dfrac{dv}{dx}
				adx &= vdv
			\end{align*}
			Ovvero se dalla posizione $x$ dove un punto possiede una velocit\`a $v$ e un'accelerazione $a$ si ha uno spostamento $dx$, allora il punto subisce una variazione di velocit\`a $dv$.
			Integrando:
			\begin{align*}
				\int_{t_0}^t a(x)dx &= \int_{v_0}^{v} vdv\\
					       &= \dfrac{1}{2}v^2 -\dfrac{1}{2}v_0^2
			\end{align*}
			Dove $v_0$ \`e la velocit\`a in $x_0$.
			Questo permette il calcolo della variazione di velocit\`a nel passaggio dalla posizione $x_0$ a $x$.

				\subparagraph{Moto uniformemente accelerato}
				Nel moto uniformemente accelerato:
				$$v_2 = v_0^2 + 2a(x-x_0)$$

	\subsection{Moto verticale di un corpo}
	Trascurando l'attrito con l'aria un corpo lasciato libero di cadere in vicinanza della superficie terrestre si move verso il basso con una accelerazione costante $g=9.8\frac{m}{s^2}$.
	Il moto \`e pertanto rettilineo uniformemente accelerato.

		\subsubsection{Sistema di riferimento}
		Il sistema di riferimento ha origine al suolo e l'asse delle $x$ rivolto verso l'alto.
		In questo sistema pertanto $a=-g=-9.8\frac{m}{s^2}$.

		\subsubsection{Caduta da un'altezza con velocit\`a iniziale nulla}
		Nel caso della caduta da un'altezza $h$ con velocit\`a iniziale nulla si nota come inizialmente:
		\begin{multicols}{3}
			\begin{itemize}
				\item $x_0 = h$.
				\item $v_0 = 0$.
				\item $t = t_0 = 0$.
			\end{itemize}
		\end{multicols}

			\paragraph{Velocit\`a}
			Dalla dipendenza della velocit\`a dal tempo nel moto uniformemente accelerato si ottiene:
			$$v(t) = -gt$$
			E si nota come la velocit\`a aumenta in modulo durante la caduta.

			\paragraph{Posizione}
			Osservando la dipendenza della posizione dal tempo nel moto uniformemente accelerato si ottiene:
			$$x = h -\frac{1}{2}gt^2$$

			\paragraph{Tempo di arrivo al suolo}
			Il tempo di arrivo al suolo, dove $x=0$ \`e:
			$$t = \sqrt{\dfrac{2h}{g}}$$

			\paragraph{Velocit\`a in funzione della posizione}
			Notando la velocit\`a in funzione della posizione nel moto uniformemente accelerato si ottiene:
			$$v^2=2g(h-x)$$

			\paragraph{Velocit\`a di arrivo al suolo}
			Il corpo arriva al suolo con una velocit\`a:
			$$v=\sqrt{2gh}$$

		\subsubsection{Caduta da un'altezza con velocit\`a iniziale non nulla}
		Nel caso della caduta da un'altezza $h$ con velocit\`a iniziale non nulla si nota come inizialmente:
		\begin{multicols}{3}
			\begin{itemize}
				\item $x_0=h$.
				\item $v_0=v_i$.
				\item $t=t_0=0$.
			\end{itemize}
		\end{multicols}

			\paragraph{Dipendenza della velocit\`a dal tempo}
			$$v(t) = -v_i -gt$$

			\paragraph{Legge oraria}
			$$x=h-v_it-\dfrac{1}{2}gt^2$$

			\paragraph{Dipendenza della velocit\`a dalla posizione}
			$$t(x) = \dfrac{-v_i+\sqrt{v_i^2+2g(h-x)}}{g}$$

			\paragraph{Tempo di caduta}
			$$t_c = \dfrac{-v_i+\sqrt{v_i^2+2gh}}{g}$$

			\paragraph{Velocit\`a di caduta}
			$$v_c^2=v)i^2+2gh$$

		\subsubsection{Lancio del punto verso l'alto partendo dal suolo}
		Nel caso di un lancio del punto verso l'alto si nota come inizialmente:
		\begin{multicols}{3}
			\begin{itemize}
				\item $x_0=0$.
				\item $v_0 = v_2 > 0$.
				\item $t=t_0=0$.
			\end{itemize}
		\end{multicols}

			\paragraph{Velocit\`a}
			$$v=v_2 - gt$$

			\paragraph{Legge oraria}
			$$x = v_2t-\dfrac{1}{2}gt^2$$

			\paragraph{Punto pi\`u alto}
			Il punto raggiunge la posizione pi\`u alta al tempo:
			$$t_M = \dfrac{v_2}{g}$$
			E nella posizione:
			$$x_M=x(t_M)=\dfrac{v_2^2}{2g}$$

			\paragraph{Discesa}
			Per $t\ge t_M$ si \`e nella situazione del primo esempio: punto che cade da un'altezza $x_M$ con velocit\`a iniziale nulla.
			Pertanto:
			$$t_s = \sqrt{2x_M}{g}=t_M$$
			E la durata complessiva del moto \`e pertanto:
			$$2t_M = \dfrac{2v_2}{g}$$
			Ricavando $t(x)$ dalla legge oraria e da $v(x)$ si ha:
			\begin{align*}
				t(x) &= \dfrac{v_2\pm \sqrt{v_2^2 - 2gx}}{g}\\
				     &= t_M \pm \sqrt{t_M^2 - \dfrac{2x}{g}}\\
				v(x) &= \pm \sqrt{v_2^2 - 2gx}
			\end{align*}

	\subsection{Moto armonico semplice}
	IL moto armonico semplice lungo un asse rettilineo \`e un moto vario la cui legge oraria \`e data dalla relazione:
	$$x(t) = A \sin(\omega t + \phi)$$
	Dove:
	\begin{multicols}{2}
		\begin{itemize}
			\item $A$ ampiezza del moto.
			\item $\omega t + \phi$ fase del moto.
			\item $\omega$ pulsazione.
			\item $\phi$ fase iniziale.
		\end{itemize}
	\end{multicols}

		\subsubsection{Caratteristiche}
		Essendo i valori estremi della funzione seno $+1$ e $-1$ il punto percorre un segmento di ampiezza $2A$ con centro nell'origine, con uno spostamento massimo da essa $A$.
		Al tempo $t=0$ occupa $x(0)=A\sin\phi$.
		Date le costanti $A$ e $\phi$ si determina la posizione iniziale del punto, che si trova a $t=0$ nell'origine solo se $\phi=\{0, \pi\}$.

		\subsubsection{Periodicit\`a}
		Essendo la funzione seno periodica con periodo $2\pi$ il moto risulta periodico e descrive oscillazioni di ampiezza $A$ rispetto al centro $O$ uguali tra loro e caratterizzate da una durata $T$ periodo del moto armonico.
		Si dice pertanto periodico un moto in cui intervalli di tempo uguali il punto ripassa nella stessa posizione con la stessa velocit\`a.

		\subsubsection{Determinare il periodo}
		Per determinare il periodo $T$ si considerino due tempi $t$ e $t'$ tali che $t'-t = T$.
		Per definizione $x(t')=x(t)$, pertanto dalla legge oraria le fasi nei due istanti devono differire $2\pi$.
		Si ha pertanto $\omega t' +\phi = \omega t + \phi + 2\pi$, ne segue che:
		\begin{align*}
			T &= \dfrac{2\pi}{\omega}\\
			\omega &=\dfrac{2\pi}{T}
		\end{align*}

			\paragraph{Significato di $\omega$}
			SI nota pertanco come il moto si ripete velocemente quando la pulsazione \`e grande mentre il moto \`e lento per vassi valori della pulsazione.

			\paragraph{Frequenza del moto}
			Si definisce frequenza $v$ del moto il numero di oscillazioni in un secondo:
			\begin{align*}
				v &= \dfrac{1}{T}\\
				  &= \dfrac{\omega}{2\pi}\\
				\omega &= 2\pi v
			\end{align*}
			Si noti come il periodo e la frequenza di un moto armonico sono indipendenti dall'ampiezza del moto.

			\paragraph{Classi di moti armonici}
			Fissato il valore della pulsazione si ottiene una classe di moti armonici caratterizzata dallo stesso periodo che differiscono tra loro per i diversi valori dell'ampiezza e della fase iniziale, ovvero per le condizioni iniziali.

		\subsubsection{Velocit\`a}
		La velocit\`a del punto che si muove con moto armonico si ottiene derivando $x(t)$.
		\begin{align*}
			\dot{x} &= v(t) = \dfrac{dx}{dt}\\
			  &=\omega A \cos(\omega t+\phi)\\
		\end{align*}
		La velocit\`a assume il valore massimo nel centro di oscillazione dove vale $\omega A$ e si annulla agli estremi dove si inverte il senso del moto.

		\subsubsection{Accelerazione}
		L'accelerazione del punto che si muove con moto armonico si ottiene derivando $v(t)$.
		\begin{align*}
			\ddot{x} &= a(t) = \dfrac{dv}{dt} = \dfrac{d^2x}{dt^2}\\
			       &= -\omega^2 A \sin(\omega t + \phi)\\
			       &= -\omega^2 x
		\end{align*}
		L'accelerazione si annulla nel centro di oscillazione e assume il valore in modulo massimo $\omega^2 A$ agli estremi, dove si inverte la velocit\`a.
		Si nota inoltre come sia proposizionale e d opposta allo spostamento dal centro di oscillazione.
		A parte il valore dell'ampiezza le tre funzioni mostrano lo stesso andamento temporale, si nota unicamente uno spostamento di una rispetto all'altra lungo l'asse dei tempi.
		Si nota pertanto come la velocit\`a sia sfasata di $\frac{\pi}{2}$  rispetto allo spostamento o si trova in quadratura di fase, mentre l'accelerazoine \`e sfasata di $\pi$ rispetto allo spostamento o si trova in opposizione di fase.

		\subsubsection{Condizioni iniziali}
		Le costanti $A$ e $\phi$ identificano le condizioni iniziali:
		$$ x(0) = x_0 = A \sin\phi$$
		$$v(0) = v_0 = \omega A \cos\phi$$
		Note le condizioni iniziali $x_0$ e $v_0$ si calcolano $A$ e $\phi$ come:
		$$\tan \phi = \dfrac{\omega x_0}{v_0}$$
		$$A^2 = x_0^2 + \dfrac{v_0^2+}{\omega^2}$$

		\subsubsection{Dipendenza della velocit\`a dalla posizione}
		\begin{align*}
			v(x) &= \int_{x_o}^x a(x)dx\\
			     &=-\omega^2\int_{x_0}^x xdx\\
			     &=\dfrac{1}{2}\omega^2(x_x^2-x^2)\\
			     &=\dfrac{1}{2}v^2 - \dfrac{1}{2}v_0^2
		\end{align*}
		Pertanto
		$$v^2 = v_0^2 + \omega^2(x_0^2(x_0^2-x^2)$$
		Con riferimento al centro dove $x_0 = 0$ e $v_0 = \omega A$
		$$v^2(x)=\omega^2(A^2 - x^2)$$
		Il segno di $v$ dipende dal verso di passaggio.
		Si nota pertanto come l'accelerazione \`e proporzionale allo spostamento con segno negativo $a =-\omega^2 x$.

		\subsubsection{Condizione sufficiente per un moto armonico semplice}
		Se si trova che in un moto l'accelerazione \`e proporzionale allo spostamento con costante di proporzionalit\`a negativa si dimostra che quel moto \`e armonico semplice.
		La condizione necessaria e sufficiente affinch\`e un moto sia armonico \`e:
		$$\dfrac{d^2x}{dt^2}+\omega^2x=0$$
		O equazione differenziale del moto armonico.
		Le funzioni seno e coseno e le loro combinazioni lineari sono tutte e sole le funzioni che soddisfano la condizione nel campo reale.

			\paragraph{Moto con funzione coseno}
			Queste considerazioni portano a considerare una legge del moto che utilizzi la funzione coseno.
			Si noti come le due funzioni differiscono per un termine di sfasamento $\frac{\pi}{2}$.
			Ovvero $x=A\sin(\omega t+\phi)$ e $x = A\cos(\omega t+\phi)$ rappresentano lo stesso moto, solo che il primo \`e visto a partire dall'istante $t_0$, mentre il secondo dall'istante $t_0 + \dfrac{T}{4}$.

		\subsubsection{Oscillazione}
		Se in un diverso fenomeno fisico si trova una grandezza $f$ che obbedisce a
		$$\dfrac{d^2f}{dz^2}+k^2f=0$$
		La soluzione \`e sempre:
		$$f(z)=A\sin(kz+\phi)$$
		Ovvero $f$ descrive un'oscillazione rispetto a $z$ il cui periodo dipende da $k$.

	\subsection{Moto rettilineo smorzato esponenzialmente}
	Si consideri ora un altro moto vario in cui l'accelerazione soddisfa la condizione $a = -kv$, con $k$ costante positiva.
	L'accelerazione \`e sempre contraria alla velocit\`a che deve necessariamente diminuire e varia ocn la stessa legge con cui varia la velocit\`a, ovvero:
	$$\dfrac{dv}{dt} = -kv$$
	Integrando con il metodo della separazione delle variabili:
	\begin{align*}
		&\dfrac{dv}{v} = -kdt\\
		\Rightarrow&\int_{v_0}^v\dfrac{dv}{v} = -k\int_0^t\\
		\Rightarrow &\log\dfrac{v}{v_0} = -kt
	\end{align*}
	Dove $v_0$ \`e la velocit\`a in $t=0$ e $v_0\neq 0$.
	Passando alle esponenziali:
	$$v(t) = v_0e^{-kt}$$
	La velocit\`a decresce esponenzialmente nel tempo e il punto alla fine si ferma.

		\subsubsection{Cambio della velocit\`a con la posizione}
		\begin{align*}
			&a = \dfrac{dv}{dt}=\dfrac{dv}{dx}\dfrac{dx}{dt}=\dfrac{dv}{dx}v = -kv\Rightarrow\\
			&\Rightarrow \dfrac{dv}{dx}=-k\Rightarrow\\
			&\Rightarrow dv = -kdx\Rightarrow\\
			&\Rightarrow \int_{v_0}^v dv = -k \int_0^xdx\\
		\end{align*}
		Risulta pertanto un andamento lineare decrescente
		$$v(x)=v_0-kx$$
		La velocit\`a si annulla in:
		$$x=\dfrac{v_0}{k}$$
		Dove il punto si ferma.

		\subsubsection{Legge oraria}
		La legge oraria si ricava per integrazione da $v(t)$:
		\begin{align*}
			x(t)&=x_0+\int_0^t v(t)dt=\\
			    &=\int_0^tv_0e^{-kt}dt=\\
					&=-\dfrac{v_0}{k}[e^{-kt}]_0^t=\\
					&=\dfrac{v_0}{k}(1-e^kt)
		\end{align*}

		\subsubsection{Costante di tempo}
		Si definisce costante di tempo:
		$$\tau = \dfrac{1}{k}$$
		In un intervallo di tempo pari a $\tau$ la funzione si riduce di un fattore di $e$.
		Minore il valore di $\tau$ pi\`u rapida la decrescita.

\section{Moto nel piano}
Nel caso in cui il moto sia vincolato a svolgersi su un piano la traiettoria del punto $P$ \`e in generale una linea curva.
Occorre pertanto specificare oltre al valore numerico dello spostamento la sua direzione e il verso.
Queste grandezze con caratteristiche direzionali e numeriche si dicono vettori.
Anche la velocit\`a e l'accelerazione del moto piano sono grandezze vettoriale, caratteristica vera per qualsiasi moto in $n$ dimensioni.

	\subsection{Posizione}
	La posizione del punto viene identificata da due coordinate che possono essere con riferimento ad un sistema di assi cartesiani ortogonali $x(t)$ e $y(t)$, o in termini di coordinate polari $r(t)$ e $\theta(t)$.

		\subsubsection{Relazione tra coordinate cartesiane e polari}
		Le relazioni tra coordinate cartesiane e polari sono:
		\begin{align*}
			x = r\cos\theta,\qquad&y=r\sin\theta\\
			r=\sqrt{x^2+y^2},\quad &\tan\theta=\dfrac{y}{x}\\
		\end{align*}

		\subsubsection{Identificare la posizione attravreso il raggio vettore}
		Si intende per raggio vettore del punti $P$:
		$$r(t)=OP=x(t)u_x+y(t)u_y$$
		Dove $u_x$ e $u_y$ rappresentano i versori degli assi cartesiani fissi nel tempo.
		Nota la dipendenza dal tempo di $r$ \`e individuato il moto di $P$.

		\subsubsection{Coordinata curvilinea}
		La posizione del punto lungo la traettoria pu\`o essere data da una coordinata curvilinea $s$ misurata a partire da un'origine arbitraria.
		Il valore di $s$ esprime la lunghezza della traiettoria e varia nel tempo dirante il moto: $\frac{ds}{dt}$ indica la variazione temporale della posizione lungo la traiettoria o la velocit\`a istantanea del punto, come definita nel moto rettilineo.

	\subsection{Velocit\`a vettoriale}
	Data la forma della traiettoria e la velocit\`a in cui viene percorsa si fornisce una descrizione completa del moto che pu\`o essere riassunta nella velocit\`a vettoriale.
	Considerando due posizioni occupate dal punto $P$ al tempo $t$ e al tempo $t+\Delta t$, queste sono individuate dai vettori $r(t)$ e $t(t+\Delta t)=r(t)+\Delta r$.
	Costruendo il rapporto incrementale:
	$$\dfrac{r(t+\Delta t)-r(t)}{\Delta t}=\dfrac{\delta r}{\Delta t}$$
	Si definisce la velocit\`a vettoriale:
	$$v=\lim\limits_{\Delta t\rightarrow 0}\dfrac{\Delta r}{\Delta t}=\dfrac{dr}{dt}$$
	Si nota pertanto come la velocit\`a vettoriale \`e la derivata del raggio vettore rispetto al tempo.
	Si pu\`o inoltre descrivere:
	$$dr=dsu_T$$
	In quanto l'incremento $dr$ risulta tangente alla traiettoria di $P$ in modo eguale allo spostamento infinitesimo $ds$, dove $u_T$ \`e il versore della tangente alla curva.
	Si pensa al moto come una successione di spostamenti rettilinei infinitesimi con direzione variabile.
	La direzione istantanea del moto coincide con quella della tangente alla traiettoria nel punto occupato all'istante considerato:
	$$v=\dfrac{ds}{dt}u_T=vu_T$$
	La velocit\`a vettoriale $v$ individua in ogni istante con la sua direzione e verso la direzione e il verso del moto e con il suo modulo $\frac{ds}{dt}$ la velocit\`a istantanea con cui \`e percorsa la traiettoria.

		\subsubsection{Invarianza}
		Si nota come la traiettoria del moto e la velocit\`a $vu_T$ sono caratteristiche intrinseche che non dipendono dalla scelta del sistema di riferimento.
		Spostando l'origine e ruotando gli assi la curva, la direzione, il verso e il modulo della velocit\`a restano gli stessi.
		Si parla pertanto di invarianza delle relazioni vettoriali rispetto alla scelta del sistema di riferimento.

		\subsubsection{Calcolo delle componenti della velocit\`a}

			\paragraph{Componenti cartesiane}
			Essendo $r=xu_x+yu_y$,
			\begin{align*}
				v &=\dfrac{dr}{dt}=\\
					&=\dfrac{dx}{dt}u_x+\dfrac{dy}{dt}u_y=\\
					&=v_xu_x + v_y+u_y
			\end{align*}
			La velocit\`a del punto $P$ ha componenti cartesiani $v_x$ e $v_y$ dei due moti rettilinei descritti dai punti proiezione di $P$ sugli assi cartesiani.
			Pertanto
			$$v=\sqrt{v^2_x+v_y^2}$$
			Inoltre detto $\phi$ l'angolo tra il vettore $v$ e l'asse $x$,
			$$\tan\phi = \dfrac{v_y}{v_x}$$

			\paragraph{Componenti polari}
			Introducendo $u_r$ e $u_\theta$, il versore della direzione di $r$ e il versore ortogonale alla stessa, si nota come questi cambiano direzione durante il moto.
			Il raggio vettore $r$ pu\`o essere espresso come $ru_r$, pertanto:
			\begin{align*}
				v&=\dfrac{dr}{dt}=\\
				 &=\dfrac{dr}{dt}u_r+r\dfrac{du_r}{dt}\Rightarrow\\
				v&=\dfrac{dr}{dt}u_r+r\dfrac{d\theta}{dt}u_\theta=\\
				 &=v_r+v_\theta\\
			\end{align*}
			La velocit\`a, sempre tangente alla traiettoria si scompone in due componenti: la velocit\`a radiale $v_r$ diretta lungo $r$ e di modulo $\frac{dr}{dt}$ e la velocit\`a trasversa $v_\theta$ ortogonale a $r$ e di modulo $r\frac{d\theta}{dt}$.
			$v_r$ dipende dalle variazioni del modulo del raggio vettore $v_\theta$, collegata alle variazioni di direzione dello stesso.
			Il modulo della velocit\`a \`e pertanto, per queste componenti:
			$$v=\dfrac{ds}{dt}=\sqrt{\biggl(\dfrac{dr}{dt}\biggr)^2+r^2\biggl(\dfrac{d\theta}{dt}\biggr)^2}$$

		\subsubsection{Determinare la posizione nota la velocit\`a}
		Essento $v=\frac{dr}{dt}$, per ricavare la posizione da essa si integra:
		$$r(t)=r(t_0)+\int_{t_0}^tv(t)dt$$
		L'integrazione esplicita pu\`o essere fatta ricorrendo alle componenti, applicando
		$$x(t)=x_0+\int_{t_0}^tv(t)dt$$
		Ai moti rettilinei componenti.

	\subsection{Accelerazione nel moto piano}
	L'accelerazione del moto piano deve esprimere le variazioni della velocit\`a come modulo che direzione.

		\subsubsection{Direzione}
		L'accelerazione non \`e parallela alla velocit\`a ed \`e diretta verso la concavit\`a della curva che rappresenta la traiettoria.

		\subsubsection{Definizione}
		L'accelerazione si definisce come derivata della velocit\`a rispetto al tempo:
		$$a=\dfrac{dv}{dt}=\dfrac{d^2r}{dt^2}$$
		Pertanto considerando $v=\frac{ds}{dt}u_T=vu_T$ e la derivata di un vettore:
		\begin{align*}
			a&=\dfrac{d}{dt}(vu_T)=\\
			 &=\dfrac{dv}{dt}u_T+v\dfrac{du_T}{dt}=\\
			 &=\dfrac{dv}{dt}u_T+v\dfrac{d\phi}{dt}u_N
		\end{align*}
		La prima componente parallela alla velocit\`a esprime la variazione del modulo della velocit\`a, mentre il secondo, dipendente dalla variazione di direzione della velocit\`a \`e ortogonale ad essa: $u_N$ \`e un vettore ortogonale a $u_T$ diretto verso la concavit\`a della traiettoria.
		$\frac{d\phi}{dt}$ determina quanto rapidamente cambia la direzione di $u_T$ e di $u_N$.

			\paragraph{Componente normale}
			Per esprimere la componente normale, si nota come al limite per $\Delta t\rightarrow 0$ le rette normali alla traiettoria in due punti molto vicini si incontrano in $C$ che coincide con il centro della circonferenza tangente alla traiettoria in $P$, o circonferenza osculatrice.
			Questo punto si chiama centro di curvatura della traiettoria nel punto $P$.
			L'arco di traiettoria $ds$ \`e pari a $Rd\phi$ con $R=CP$ raggio di curvatura.
			Al variare di $P$ lungo la traiettoria sia $R$ che $X$ variano.
			Si nota:
			$$\dfrac{d\phi}{dt}=\dfrac{d\phi}{ds}\dfrac{ds}{dt}=\dfrac{1}{R}v$$
			Sostituendo nell'espressione dell'accelerazione trovata prima:
			$$a=\dfrac{dv}{dt}u_T+\dfrac{v^2}{R}u_N=a_T+a_N$$
			Il modulo pertanto vale:
			$$a=\sqrt{a^2_T+a^2_N}=\sqrt{\biggl(\dfrac{dv}{dt}\biggr)^2+\dfrac{v^4}{R^2}}$$
			Le due componenti si chiamano accelerazione tangenziale e accelerazione normale o centripeta.

		\subsubsection{Moto curvilineo uniforme}
		In un moto curvilineo vario entrambe le comopnenti sono diverse da zero, me il moto \`e curvilineo uniforme $a_T$ \`e nulla, mentre il moto \`e rettilineo vario \`e nulla $a_N$, nel moto rettilineo uniforme sono entrambe nulle.
		Ovvero:
		\begin{multicols}{2}
			\begin{itemize}
				\item $a_T\neq 0$ moto vario.
				\item $a_N\neq 0$ moto curvilineo.
			\end{itemize}
		\end{multicols}

		\subsubsection{Componenti cartesiane}
		Le componenti cartesiane dell'accelerazione sono le accelerazioni dei due moti rettilinei proiezioni sugli assi del moto di $P$ lungo la traiettoria curva:
		\begin{align*}
			a&=\dfrac{dv}{dt}=\\
			 &=\dfrac{dv_x}{dt}u_x+\dfrac{dv_y+_dt+u_y}=\\
			 &=\dfrac{d^2x}{dt^2}u_x+\dfrac{d^2y}{dt^2}u_y=\\
			 &=a_xu_x+a_yu_y
		\end{align*}
		Detto $\phi$ l'angolo che $u_T$ forma con $u_x$, si deduce che:
		$$a_x=\dfrac{dv}{dt}\cos\phi - \dfrac{v^2}{R}\sin\phi$$
		$$a_y=\dfrac{dv}{dt}\sin\phi - \dfrac{v^2}{R}\cos\phi$$
		Dalle componenti tangenziale e centripeta si ricavano le cartesiano risolvendo il sistema lineare nelle incognite $\frac{dv}{dt}$ e $\frac{v^2}{R}$.

		\subsubsection{Componenti polari}
		Considerando che $u_r$ e $u_\theta$ non sono fissi e $v=\frac{dr}{dt}u_r+\frac{d\theta}{t}u_\theta$
		\begin{align*}
			a&=\dfrac{dv}{dt}=\\
			 &=\dfrac{d}{t}\biggl(\dfrac{dr}{dr}u_r+r\dfrac{d\theta}{dt}u_\theta\biggr)=\\
			 &=\dfrac{d^2r}{dt^2}u_r+\dfrac{dr}{dt}\dfrac{du_r}{dt}+\dfrac{dr}{dt}\dfrac{d\theta}{dt}u_\theta+r\dfrac{d^2\theta}{dt^2}u_\theta+r\dfrac{d\theta}{dt}\dfrac{dt_\theta}{dt}
		\end{align*}
		Considerando che $\frac{du_r}{dt}=\frac{d\theta}{dt}u_\theta$ e che $\frac{du_\theta}{dt}=-\frac{d\theta}{dt}u_r$ si nota che per una variazione positiva di $\theta$ $du_\theta$ \`e opposto a $u_r$  e si ha:
		$$a=\biggl[\dfrac{d^2r}{dt^2}-r\biggl(\dfrac{d\theta}{dt}\biggr)^2\biggr]u_r+\biggl[2\dfrac{dr}{dt}\dfrac{d\theta}{dt}+r\dfrac{d^2\theta}{dt^2}\biggr]u_\theta$$
		Da cui:
		$$a=\biggl[\dfrac{d^2r}{dt^2}-r\biggl(\dfrac{d\theta}{dt}\biggr)^2\biggr]u_r+\biggl[\dfrac{1}{r}\dfrac{d}{dt}\biggl(r^2\dfrac{d\theta}{dt}\biggr)\biggr]u_theta$$
		Il primo termine rappresenta l'accelerazione radiale e il secondo l'accelerazione trasversa.
		Anche $a_r$ e $a_\theta$ si possono mettere in relazione con $a_x$ e $a_y$ o $a_T$ e $a_N$

		\subsubsection{Valore della velocit\`a}
		Nota l'accelerazione e il valore della velocit\`a all'istante $t_0$ la velocit\`a in un istante $t$ \`e data da:
		$$v(t)=v(t_0)+\int_{t_0}^ta(t)dt$$

	\subsection{Moto circolare}
	Si dice moto circolare un moto piano la cui traiettoria \`e rappresentata da una circonferenza.
	L'accelerazione centripeta \`e sempre diversa da zero, pertanto agisce una forza centripeta diretta verso il centro della circonferenza.
	Nel moto circolare uniforme la velocit\`a \`e costante in modulo e l'accelerazione tangente \`e nulla per cui $a=a_N$, se invece il modulo della velocit\`a cambia nel tempo il moto circolare non \`e uniforme e $a_T$ \`e diversa da zero: la direzione dell'accelerazione non passa per il centro della circonferenza.
	In quest'ultimo caso oltre alla forza centripeta agisce anche una forza tangenziale.

		\subsubsection{Descrizione}
		Il moto circolaer pu\`o essere descritto facendo riferimento allo spazio percorso sulla circonferenza $s(t)$ o utilizzando l'angolo $\theta(t)$ sotteso all'arco $s(t)$ con $\theta(t)=\frac{s(t)}{R}$.
		Assumere come variabile $\theta(t)$ vuol dire porsi in un sistema di coordinate polari di centro in $O$ in cui il moto avviene con $r(t)=R$ costante.
		La rappresentazione in coordinate cartesiane \`e legata a $\theta(x)$:
		$$x(t)=R\cos\theta(t)$$
		$y(t)=R\sin\theta(t)$

		\subsubsection{Velocit\`a angolare}
		Si \`e interessati alle variazioni dell'angolo nel tempo e si definisce la velocit\`a angolare come la derivata dell'angolo rispetto al tempo:
		$$\omega=\dfrac{d\theta}{dt}=\dfrac{1}{R}\dfrac{ds}{dt}=\dfrac{v}{R}$$
		Si nota come la velocit\`a angolare \`e proporzionale alla velocit\`a con cui \`e descritta la circonferenza.
		Nel moto circolare la velocit\`a radiale \`e identicamente nulla in quanto il raggio vettore \`e costante in modulo e la velocit\`a trasversale coincide con la velocit\`a: da $v_\theta=r\frac{d\theta}{dt}$ si trova che $v=R\omega$.

		\subsubsection{Moto circolare uniforme}
		Nel moto circolare uniforme $v$ e $\omega$ sono costanti e le leggi orarie con riferimento alle due variabili sono:
		$$s(t)=s_0+vt\qquad s= s_o\ per\ t=0$$
		$$\theta(t)=\theta_0+\omega t\qquad \theta=\theta_0\ per\ t = 0$$
		Si ricordi come il termine uniforme significa esclusivamente costanza del modulo della velocit\`a: il moto circolare uniforme \`e un moto accelerato con accelerazione costante ortogonale alla traiettoria:
		$$a=a_N=\dfrac{v^2}{R}=\omega^2R$$

			\paragraph{Periodicit\`a}
			Si nota come si tratta di un moto periodico con periodo
			$$T=\dfrac{2\pi R}{v}=\dfrac{2\pi}{\omega}$$
			Corrispondente al tempo necessario per compiere un giro completo.

			\paragraph{Moti proiettati sugli assi cartesiani}
			I moti proiettati sugli assi cartesiani sono:
			$$x=R\cos\theta=R\cos(\omega t+\theta_0)$$
			$$y=R\sin\theta=R\sin(\omega t+\theta_0)$$
			Ovvero due moti armonici di uguale ampiezza e fase iniziale sfasati di $\frac{\pi}{2}$ e con periodo coincidente con quello del moto circolari uniforme.
			La velocit\`a angolare \`e uguale alla pulsazione numericamente.

		\subsubsection{Moto circolare non uniforme}
		Nel caso del moto circolare non uniforme si deve considerare l'accelerazione tangenziale $a_T=\frac{dv}{dt}$.

			\paragraph{Accelerazione angolare}
			Essendo anche $\omega$ variabile si definisce l'accelerazione angolare:
			\begin{align*}
				a&=\dfrac{d\omega}{dt}=\\
				 &=\dfrac{d^2\theta}{dt^2}=\\
				 &=\dfrac{1}{R}\dfrac{dv}{dt}=\\
				 &=\dfrac{a_T}{R}\\
			\end{align*}

			\paragraph{Determinare la velocit\`a angoalre dall'accelerazione}
			Nota $a(t)$ si pu\`o integrare ottenendo:
			$$\omega(t)=\omega_0+\int_{t_0}^ta(t)dt$$
			$$\theta(t)=\theta_0+\int_{t_0}^t\omega(t)dt$$

			\paragraph{Determinare l'incremento della velocit\`a angolare in corrispondenza dell'incremento angolare}
			Nota $a(\theta)$ si pu\`o calcolare l'incremento della velocit\`a angolare in corrispondenza all'incremento angolare $\theta-\theta_0$.
			\begin{align*}
				a&=\dfrac{d\omega}{dt}=\dfrac{d\omega}{d\theta}\dfrac{d\theta}{dt}=\\
				 &=\omega\dfrac{d\omega}{d\theta}\Rightarrow\\
				\Rightarrow&ad\theta=\omega d\omega\Rightarrow\\
				\Rightarrow&\int_{\theta_0}^\theta a(\theta)d\theta=\dfrac{1}{2}\omega^2-\dfrac{1}{2}\omega_0^2
			\end{align*}

			\paragraph{Moto circolare uniformemente accelerato}
			Il moto circolare uniformemente accelerato \`e un moto in cui $a$ \`e costante, o $a_T$ costante.
			Ponendo $t_0=0$:
			$$\omega=\omega_0+at$$
			$$\theta=\theta_0+\omega_0t+\dfrac{1}{2}at^2$$
			L'accelerazione centripeta vale:
			$$a_N=\omega^2R=(\omega_0+at)^2R$$

		\subsubsection{Notazione vettoriale}
		Si ampli il concetto di velocit\`a angolare del moto circolare mostrando come si possono associare ad esso caratteristiche vettoriali.
		Si definisce velocit\`a angolare il vettore $\omega$ con modulo $\omega=\frac{d\theta}{dt}$, direzione perpendicolare al piano in cui giace la circonferenza e verso tale che all'estremo del vettore $\omega$ il moto appaia antiorario.
		Risulta evidente che:
		$$v=\omega\times r$$
		Nel caso $\omega$ sia applicata al centro della circonferenza $r=R$, ma questa equazione risulta valida se $\omega$ \`e applicata in un qualsiasi altro punto dell'asse di rotazione, ovvero la retta ortigonale al piano del moto e passante per il centro della circonferenza.
		Dato $\omega$ si individua l'asse di rotazione e il piano del moto circolare, il verso di percorrenza della circonferenza e come varia l'angolo nel tempo.
		Da $\omega$ per derivazione rispetto al tempo si ottiene il vettore accelerazione angolare $a$, parallelo a $\omega$ e verso determinato dalla variazione del modulo di $\omega$ e modulo $a=\frac{d\omega}{dt}$.

			\paragraph{Accelerazione del moto circolare}
			Tramite $A$ e $\omega$ si esprime l'accelerazione del moto circolare:
			\begin{align*}
				a&=\dfrac{dv}{dt}=\\
				 &=\dfrac{d}{dt}(\omega\times r)=\\
				 &=\dfrac{d\omega}{dt}\times r + \omega\times\dfrac{dr}{dt}\Rightarrow\\
				\Rightarrow&a=a\times r+\omega\times v\\
			\end{align*}
			Si nota come il pirmo termine $a\times r$ \`e l'accelerazione tangenziale $a_T$ di modulo $aR$, mentre il secondo $\omega\times v$ \`e l'accelerazione centripeta $a_N$ di modulo $\omega^2R$.
			Nel moto circolare uniforme $\omega$ \`e un vettore costante anche un modulo, $a$ \`e nulla e $a=a_N=\omega\times v$.

			\paragraph{Propriet\`a di $r$}
			Il vettore $r$ applicato in $O$ ha modulo costante e descrive un moto rotatorio attorno all'asse di rotazione, ovvero alla direzione di $\omega$ formando un angolo $\phi$ costante con l'asse stesso.
			La sua derivata $\frac{dr}{dt}$ si pu\`o scrivere $\omega\times r$.
			Anche il vettroe $v$ descrive una rotazione intorno a $\omega$ con cui forma l'angolo $\phi=\frac{\pi}{2}$ e la sua derivata $\frac{dv}{dt}$ si pu\`o scrivere $\omega\times v$.
			Al moto di rotazione di un asse rispetto ad un altro asse fisso con cui forma un angolo costante e ha un punto in comune si d\`a il nome di moto di precessione.
			Dato un vettore di modulo costante $A$ che descrive un moto di precessione con velocit\`a angolare $\omega$, la sua derivata temporale pu\`o essere sempre scritta:
			$$\dfrac{dA}{dt}=\omega\times A$$
			Che risulta ortogonale a $A$.
			Inoltre in modulo
			$$dA=A\sin\phi d\theta$$
			$$\dfrac{dA}{dt}=A\sin\phi\dfrac{d\theta}{dt}=\omega A\sin\phi=|\omega\times A|$$

	\subsection{Moto parabolico dei corpi}
	Si analizza il moto nel vuoto di un punto $P$ lanciato dall'origine $O$ con velocit\`a iniziale $v_0$ formante un angolo $\alpha$ con l'asse delle ascisse.
	Si vuole calcolare la traiettoria, la massima altezza raggiunta e la posizione $G$ in cui il punto ricade su $x$ o la gittata $OG$.

		\subsubsection{Caratterizzazione}
		Il moto \`e caratterizzato da un'accelerazione costante $a=g=-gu_x$ e le condizioni inziali sono $r=0$ e $v=v_0$ al tempo $t=0$.
		Si nota come
		$$v(t)=v_0+\int_{t_0}^ta(t)dt=v_0-gtu_x$$

		\subsubsection{Velocit\`a}
		La velocit\`a si trova nel piano individuato dai vettori costanti $v_\theta$ e $g$, il piano $x,y$.
		Essendo $v_\theta=v_0\cos t au_x+v_\theta\sin au_y$:
		$$v(t)=v_\theta\cos au_y+(v_\theta\sin a-gt)u_x$$

		\subsubsection{Leggi orarie dei moti proiettati}
		La velocit\`a dei moti proiettati sugli assi sono $v_y=v_o\cos\alpha$ e $v_x=v_0\sin\alpha-gt$.
		Pertanto le leggi orarie dei moti proiettati sugli assi sono:
		$$x=v_0\cos t\alpha t$$
		$$y=v_0\sin\alpha t-\dfrac{1}{2}gt^2$$
		Se sull'asse $x$ il moto \`e uniforme sull'asse $y$ \`e uniformemente accelerato.

		\subsubsection{Traiettoria}
		La traiettoria viene ricavata eliminando il tempo tra $x(t)$ e $y(t)$ e ottenendo la funzione:
		$$y(t):t=\dfrac{x}{v_0}\cos\alpha$$
		$$y(x)=x\tan\alpha-\dfrac{g}{2v_0^2\cos^2\alpha}x^2$$
		Ovvero l'equazione di una parabola.

		\subsubsection{Direzione del moto}
		La direzione del moto in funzione del tempo o della coordinata $x$ pu\`o essere caratterizzata dall'angolo $\phi$ che il vettore velocit\`a forma con l'asse orizzontale:
		\begin{align*}
			\tan\phi&=\dfrac{v_y}{v_x}=\\
			     &=\tan\alpha-\dfrac{g}{v_0\cos\alpha}t=\\
					 &=\tan\alpha-\dfrac{g}{v_0^2\cos^2\alpha}x
		\end{align*}
		Per calcolare la gittata $OG$ si impne $y(x)=0$ e si ottengono due soluzione $x=0$ e:
		\begin{align*}
			x_G&=\dfrac{2v_0^2\cos^2\alpha\tan\alpha}{g}=\\
			   &=\dfrac{2v_0^2\cos\alpha\sin\alpha}{g}=\\
				 &=2X_M
		\end{align*}
		Dove $X_M=v_0^2\frac{\cos\alpha\sin\alpha}{g}$ \`e la coordinata dal punto di mezzo del segmento $OG$ e ascissa del punto di massima altezza.

		\subsubsection{Altezza massima}
		L'altezza massima raggiunta \`e pertanto:
		$$y(X_M)=Y_M=\dfrac{v_0^2\sin^2\alpha}{2g}$$
		L'altezza massima si pu\`o ottenere annullando $\frac{dy}{dx}$, ovvero imponendo
		$$\tan\alpha-(\dfrac{g}{v_0^2}\cos^2\alpha)x=0$$
		Si ricava da questo l'ascissa del massimo $X_M$ e si calcola $y(X_M)$.
		Un altro modo \`e sfruttare il fatto che nel punto di massima altezza la velocit\`a \`e orizzontale, pertanto $u_y=0$< ovvero $t=t_M=v_0\sin\frac{\alpha}{g}$, sostituendo $x(t)$ e $y(t)$ si trovano $X_M$ e $Y_M$.

		\subsubsection{Angolo di lancio per la gittata massima}
		L'angolo di lancio per cui si ha la gittata massima si ottiene con la condizione $\frac{dx_G}{d\alpha}=0$< ovvero $\frac{2v_0^2}{g}(-\sin^2\alpha+\cos^2\alpha)=0$ e risulta $\alpha=45\si{\degree}$

		\subsubsection{Tempo totale di volo}
		Il tempo totale di volo $t_G$ \`e pari al tempo impiegato a percorere $OG$ con velocit\`a costante $v_x=v_0\cos\alpha$:
		$$t_G=2\dfrac{X_M}{v_0}\cos\alpha=2v_0\sin\dfrac{\alpha}{g}=2t_M$$
		Si nota come $t_G$ coincide con il tempo necessario per salire a $Y_M$ e tornare al suolo.
		Si nota come nella posizione $G$ la velocit\`a \`e la stessa in modulo che alla partenza ma simmetricamente rispetto a $x$.
		$$u_x(t_G)=v_0\cos\alpha$$
		$$u_y(t_G)=-v_o\sin\alpha$$
		$$\tan\phi=-\tan\alpha$$
