\chapter{Cinematica del punto}

\section{Introduzione}
La meccanica riguarda lo studio del moto di un corpo: spiega la relazione tra le cause che lo generano e le sue caratteristiche, esprimendola con leggi quantitative.

	\subsection{Punto materiale}
	Un punto materiale o particella \`e un corpo privo di dimensioni: le sue dimensioni sono trascurabili rispetto a quelle dello spazio in cui pu\`o muoversi o degli altri corpi con cui pu\`o interagire.

	\subsection{Movimenti di un corpo esteso}
	\begin{multicols}{2}
		\begin{itemize}
			\item Traslazione: il corpo esteso si muove come un punto materiale.
			\item Rotazioni.
			\item Vibrazioni.
		\end{itemize}
	\end{multicols}

	\subsection{Cinematica}
	Si intende per cinematica una parte della meccanica che studia il moto senza considerare le forze che entrano in gioco.

	\subsection{Determinare il moto}
	Il moto di un punto materiale \`e determinato se \`e nota la sua posizione in funzione del tempo in un determinato sistema di riferimento.

		\subsubsection{Sistema di riferimento cartesiano}
		In un sistema di riferimento cartesiano il la posizione di un corpo \`e data dalle sue coordinate $x(t)$, $y(t)$, $z(t)$, espresse in funzione del tempo.
		Altri sistemi di riferimento fanno uso delle coordinate polari.

	\subsection{Traiettoria}
	La traiettoria \`e il luogo dei punti occupati successivamente dal punto in movimento.
	Costituisce una curva continua nello spazio.

	\subsection{Grandezze fondamentali}
	Nella cinematica le grandezze fondamentali sono:
	\begin{multicols}{2}
		\begin{itemize}
			\item Spazio.
			\item Velocit\`a.
			\item Accelerazione.
			\item Tempo o la variabile indipendente.
		\end{itemize}
	\end{multicols}

	\subsection{Quiete}
	La quiete \`e un tipo di moto in cui le coordinate rimangono costanti, pertanto velocit\`a ed accelerazione sono nulle.

\section{Moto rettilineo}

	\subsection{Descrizione}
	Il moto rettilineo si svolge lungo una retta su cui vengono fissati arbitrariamente un'origine e un verso.
	Il moto del punto pu\`o essere descrivibile tramite una coordintata $x(t)$.

	\subsection{Rappresentazione}
	Le misure ottenute da un'osservazione di un moto rettilineo per tempo e spazio possono essere rappresentate in un sistema a due assi cartesiani: sulle ordinate i valori di $x$ e su quello delle ascisse il tempo $t$ corrispondente.
	Questo viene detto diagramma orario.

	\subsection{Velocit\`a}

		\subsubsection{Velocit\`a media}
		Se al tempo $t=t_1$ il punto si trova nella posizione $x=x_1$ e al tempo $t=t_2$ nella posizione $x=x_2$, $\Delta x = x_2 - x_1$ rappresenta lo spazio percorso nell'intervallo di tempo $\Delta t = t_2 - t_1$.
		La rapidit\`a con cui avviene lo spostamento viene caratterizzata dalla velocit\`a media:
		$$v_m = \dfrac{\Delta x}{\Delta t} = \dfrac{x_2 - x_1}{t_2 - t_1}$$

		\subsubsection{Velocit\`a istantanea}
		Per ricavare informazioni riguardo le caratteristiche del moto si pu\`o suddividere $\Delta x$ in numerosi piccoli intervalli $(\Delta x)_i$ percorsi in altrettanti piccoli intervalli di $\Delta t$ $(\Delta t)_i$.
		Si nota come le corrispondenti velocit\`a medie sono $v_i = \frac{(\Delta x)_i}{(\Delta t)_i}$, diverse tra di loro e da $v_m$.
		Questo avviene in quanto in un generico moto rettilineo la velocit\`a non \`e costante nel tempo.
		Suddividendo $\Delta x$ in un numero elevatissimo di intervallini $dx$ percorsi nel tempo $dt$ si pu\`o definire la velocit\`a istantanea ad un istante $t$ del punto in movimento come:
		$$v = \lim\limits_{\Delta t \rightarrow 0} \dfrac{\Delta x}{\Delta t} = \dfrac{dx}{dt}$$
		La velocit\`a istantanea rappresenta pertanto la rapidit\`a di variazione temporale della posizione nell'istante $t$ considerato.
		Il segno indica il verso del moto sull'asse.
		Pu\`o inoltre essere espressa come funzione del tempo $v(t)$.

			\paragraph{Moto rettilineo uniforme}
			Si intende per moto rettilineo uniforme un tipo di moto rettilineo in cui la velocit\`a \`e costante.

			\paragraph{Ottenere la velocit\`a}
			Nota la legge oraria $x(t)$ si pu\`o otenere la velocit\`a istantanea con l'operazione di derivazione.

			\paragraph{Ottenere la legge oraria}
			Nota la dipendenza del tempo della velocit\`a istantanea $v(t)$ si pu\`o ottenere la legge oraria $x(t)$.
			Supponendo che il punto si trovi in $x$ al tempo $t$ e nella posizione $x+dx$ in $t+dt$ da $v=\frac{dx}{dt}$ si nota come lo spostamento infinitesimo $dx$ \`e uguale al prodotto del tempo $dt$ impiegato a percorrerlo per il valore della velocit\`a al tempo $t:dx=v(t)dt$, qualunque sia la dipendenza della velocit\`a dal tempo.
			Lo spostamento complessivo sulla retta su cui si muove il punto in un intervallo finito $\Delta t = t - t_0$ \`e dato dalla somma di tutti i successivi valori $dx$.
			Si utilizza l'operazione di integrazione:
			\begin{align*}
				&\Delta x = \int_{x_0}^x dx = \int_{t_0}^t v(t)dt\\
				&x - x_0 = \int_{t_0}^t v(t)dt\\
				&x = x_0 +\int_{t_0}^tv(t)dt
			\end{align*}
			Si ottiene pertanto la relazione generale che permette il calcolo dello spazio percorso nel moto rettilineo:
			$$x(t) = x_0 + \int_{t_0}^t v(t)dt$$
			Dove $x_0$ rappresenta la posizione iniziale del punto occupata nell'istante $t_0$
			Si noti come $\Delta x$ rappresenta la somma algebrica degli spostamenti.

		\subsubsection{Relazione tra velocit\`a media e istantanea}
		Ricordando che $v_m = \frac{x-x_0}{t-t_0}$, la relazione tra velocit\`a media e istantanea:
		$$v_m = \dfrac{1}{t-t_0}\int_{t_0}^{t}v(t)dt$$

		\subsubsection{Legge oraria del moto rettilineo uniforme}
		Considerando il moto rettilineo uniforme in cui $v$ \`e costante si ha:
		\begin{align*}
			x(t) &= x_0 + v\int_{t_0}^t dt\\
			     &=x_0 + v(t-t_0)\\
			     &=x_0 + vt\qquad\qquad se\ t_0 = 0
		\end{align*}
		Si nota pertanto come nel moto rettilineo uniforme lo spazio \`e una funzione lineare del tempo e la velocit\`a istantanea coincide con la velocit\`a media.

	\subsection{Accelerazione}
	La velocit\`a $v(t)$ varia in un determinato $\Delta t$ di una quantit\`a $\delta v$.

		\subsubsection{Accelerazione media}
		Analogamente alla velocit\`a media si definisce l'accelerazione media come
		$$a_m = \dfrac{\Delta v}{\Delta t}$$

		\subsubsection{Accelerazione istantanea}
		Si definisce accelerazione istantanea come la rapidit\`a di variazione temporale della celocit\`a come:
		$$a = \dfrac{dv}{dt} = \dfrac{d^2 x}{dt^2}$$

		\subsubsection{Significato fisico dell'accelerazione}
		\begin{multicols}{2}
			\begin{itemize}
				\item $a=0$: velocit\`a costante, moto rettilineo uniforme.
				\item $a>0$: la velocit\`a cresce nel tempo.
				\item $a<0$: la velocit\`a decresce nel tempo.
			\end{itemize}
		\end{multicols}

		\subsubsection{Ottenere la velocit\`a}
		Data una $a(t)$ si ricava $v(t)$:
		\begin{align*}
			dv &=a(t)dt\\
			\Delta v &= \int_{v_0}^v dv\\
			   &=\int_{t_0}^t a(t)dt\\
		\end{align*}
		Pertanto:
		$$v(t) = v_0 + \int_{t_0}^t a(t)dt$$

		\subsubsection{Moto rettilineo uniformemente accelerato}
		Si intende per moto rettilineo uniformemente accelerato un moto in cui l'accelerazione \`e costante durante il moto.

			\paragraph{Dipendenza della velocit\`a dal tempo}
			La dipendenza della velocit\`a dal tempo \`e lineare:
			\begin{align*}
				v(t) &=v_0+a(t-t_0)\\
				v(t) &=v_0+at\qquad\qquad se\ t_0 = 0
			\end{align*}

			\paragraph{Dipendenza della posizione dal tempo}
			Lo spazio \`e una funzione quadratica del tempo:
			\begin{align*}
				x(t) &= x_0 +\int_{t_0}^t [v_0 + a(t-t_0)]dt\\
				     &= x_0 + \int_{t_0}^t v_0dt + \int_{t_0}^t a(t-t_0)dt\\
				x(t) &= x_0 + v(t-t_0) +\dfrac{1}{2}a(t-t_0)^2\\
				     &= x_0 + v_0t +\dfrac{1}{2}at^2\qquad\qquad se\ t_0 = 0
			\end{align*}

			\paragraph{Dipendenza dell'accelerazione dalla posizione}
			Nota la dipendenza dell'accelerazione dalla posizione, ovvero $a(x)$ si pu\`o ricavare il valore della velocit\`a in ogni posizione $x$ o $v(x)$.
			Questo avviene considerando le funzioni di funzione.
			Se ad un istante $t$ il punto occupa una posizione $x$ con velocit\`a $v$ e accelerazione $a$ si possono pensare come funzioni della posizione e
			$$v(t) = v[x(t)]$$
			$$a(t) = a[x(t)]$$
			Derivando la prima rispetto al tempo e sfruttando la regola di derivazione delle funzioni di funzioni:
			\begin{align*}
				a[x(t)] &= \dfrac{d}{dt}v[x(t)]\\
				        &= \dfrac{d}{dt}\dfrac{d}{dt}\dfrac{dv}{dx}\dfrac{dx}{dt}\\
					&= \dfrac{dv}{dx}\dfrac{dx}{dt}\\
				a  	&= \dfrac{dv}{dx}
				adx &= vdv
			\end{align*}
			Ovvero se dalla posizione $x$ dove un punto possiede una velocit\`a $v$ e un'accelerazione $a$ si ha uno spostamento $dx$, allora il punto subisce una variazione di velocit\`a $dv$.
			Integrando:
			\begin{align*}
				\int_{t_0}^t a(x)dx &= \int_{v_0}^{v} vdv\\
					       &= \dfrac{1}{2}v^2 -\dfrac{1}{2}v_0^2
			\end{align*}
			Dove $v_0$ \`e la velocit\`a in $x_0$.
			Questo permette il calcolo della variazione di velocit\`a nel passaggio dalla posizione $x_0$ a $x$.

				\subparagraph{Moto uniformemente accelerato}
				Nel moto uniformemente accelerato:
				$$v_2 = v_0^2 + 2a(x-x_0)$$

	\subsection{Moto verticale di un corpo}
	Trascurando l'attrito con l'aria un corpo lasciato libero di cadere in vicinanza della superficie terrestre si move verso il basso con una accelerazione costante $g=9.8\frac{m}{s^2}$.
	Il moto \`e pertanto rettilineo uniformemente accelerato.

		\subsubsection{Sistema di riferimento}
		Il sistema di riferimento ha origine al suolo e l'asse delle $x$ rivolto verso l'alto.
		In questo sistema pertanto $a=-g=-9.8\frac{m}{s^2}$.

		\subsubsection{Caduta da un'altezza con velocit\`a iniziale nulla}
		Nel caso della caduta da un'altezza $h$ con velocit\`a iniziale nulla si nota come inizialmente:
		\begin{multicols}{3}
			\begin{itemize}
				\item $x_0 = h$.
				\item $v_0 = 0$.
				\item $t = t_0 = 0$.
			\end{itemize}
		\end{multicols}

			\paragraph{Velocit\`a}
			Dalla dipendenza della velocit\`a dal tempo nel moto uniformemente accelerato si ottiene:
			$$v(t) = -gt$$
			E si nota come la velocit\`a aumenta in modulo durante la caduta.

			\paragraph{Posizione}
			Osservando la dipendenza della posizione dal tempo nel moto uniformemente accelerato si ottiene:
			$$x = h -\frac{1}{2}gt^2$$

			\paragraph{Tempo di arrivo al suolo}
			Il tempo di arrivo al suolo, dove $x=0$ \`e:
			$$t = \sqrt{\dfrac{2h}{g}}$$

			\paragraph{Velocit\`a in funzione della posizione}
			Notando la velocit\`a in funzione della posizione nel moto uniformemente accelerato si ottiene:
			$$v^2=2g(h-x)$$

			\paragraph{Velocit\`a di arrivo al suolo}
			Il corpo arriva al suolo con una velocit\`a:
			$$v=\sqrt{2gh}$$

		\subsubsection{Caduta da un'altezza con velocit\`a iniziale non nulla}
		Nel caso della caduta da un'altezza $h$ con velocit\`a iniziale non nulla si nota come inizialmente:
		\begin{multicols}{3}
			\begin{itemize}
				\item $x_0=h$.
				\item $v_0=v_i$.
				\item $t=t_0=0$.
			\end{itemize}
		\end{multicols}

			\paragraph{Dipendenza della velocit\`a dal tempo}
			$$v(t) = -v_i -gt$$

			\paragraph{Legge oraria}
			$$x=h-v_it-\dfrac{1}{2}gt^2$$

			\paragraph{Dipendenza della velocit\`a dalla posizione}
			$$t(x) = \dfrac{-v_i+\sqrt{v_i^2+2g(h-x)}}{g}$$

			\paragraph{Tempo di caduta}
			$$t_c = \dfrac{-v_i+\sqrt{v_i^2+2gh}}{g}$$

			\paragraph{Velocit\`a di caduta}
			$$v_c^2=v)i^2+2gh$$

		\subsubsection{Lancio del punto verso l'alto partendo dal suolo}
		Nel caso di un lancio del punto verso l'alto si nota come inizialmente:
		\begin{multicols}{3}
			\begin{itemize}
				\item $x_0=0$.
				\item $v_0 = v_2 > 0$.
				\item $t=t_0=0$.
			\end{itemize}
		\end{multicols}

			\paragraph{Velocit\`a}
			$$v=v_2 - gt$$

			\paragraph{Legge oraria}
			$$x = v_2t-\dfrac{1}{2}gt^2$$

			\paragraph{Punto pi\`u alto}
			Il punto raggiunge la posizione pi\`u alta al tempo:
			$$t_M = \dfrac{v_2}{g}$$
			E nella posizione:
			$$x_M=x(t_M)=\dfrac{v_2^2}{2g}$$

			\paragraph{Discesa}
			Per $t\ge t_M$ si \`e nella situazione del primo esempio: punto che cade da un'altezza $x_M$ con velocit\`a iniziale nulla.
			Pertanto:
			$$t_s = \sqrt{2x_M}{g}=t_M$$
			E la durata complessiva del moto \`e pertanto:
			$$2t_M = \dfrac{2v_2}{g}$$
			Ricavando $t(x)$ dalla legge oraria e da $v(x)$ si ha:
			\begin{align*}
				t(x) &= \dfrac{v_2\pm \sqrt{v_2^2 - 2gx}}{g}\\
				     &= t_M \pm \sqrt{t_M^2 - \dfrac{2x}{g}}\\
				v(x) &= \pm \sqrt{v_2^2 - 2gx}
			\end{align*}

	\subsection{Moto armonico semplice}
	IL moto armonico semplice lungo un asse rettilineo \`e un moto vario la cui legge oraria \`e data dalla relazione:
	$$x(t) = A \sin(\omega t + \phi)$$
	Dove:
	\begin{multicols}{2}
		\begin{itemize}
			\item $A$ ampiezza del moto.
			\item $\omega t + \phi$ fase del moto.
			\item $\omega$ pulsazione.
			\item $\phi$ fase iniziale.
		\end{itemize}
	\end{multicols}

		\subsubsection{Caratteristiche}
		Essendo i valori estremi della funzione seno $+1$ e $-1$ il punto percorre un segmento di ampiezza $2A$ con centro nell'origine, con uno spostamento massimo da essa $A$.
		Al tempo $t=0$ occupa $x(0)=A\sin\phi$.
		Date le costanti $A$ e $\phi$ si determina la posizione iniziale del punto, che si trova a $t=0$ nell'origine solo se $\phi=\{0, \pi\}$.

		\subsubsection{Periodicit\`a}
		Essendo la funzione seno periodica con periodo $2\pi$ il moto risulta periodico e descrive oscillazioni di ampiezza $A$ rispetto al centro $O$ uguali tra loro e caratterizzate da una durata $T$ periodo del moto armonico.
		Si dice pertanto periodico un moto in cui intervalli di tempo uguali il punto ripassa nella stessa posizione con la stessa velocit\`a.

		\subsubsection{Determinare il periodo}
		Per determinare il periodo $T$ si considerino due tempi $t$ e $t'$ tali che $t'-t = T$.
		Per definizione $x(t')=x(t)$, pertanto dalla legge oraria le fasi nei due istanti devono differire $2\pi$.
		Si ha pertanto $\omega t' +\phi = \omega t + \phi + 2\pi$, ne segue che:
		\begin{align*}
			T &= \dfrac{2\pi}{\omega}\\
			\omega &=\dfrac{2\pi}{T}
		\end{align*}

			\paragraph{Significato di $\omega$}
			SI nota pertanco come il moto si ripete velocemente quando la pulsazione \`e grande mentre il moto \`e lento per vassi valori della pulsazione.

			\paragraph{Frequenza del moto}
			Si definisce frequenza $v$ del moto il numero di oscillazioni in un secondo:
			\begin{align*}
				v &= \dfrac{1}{T}\\
				  &= \dfrac{\omega}{2\pi}\\
				\omega &= 2\pi v
			\end{align*}
			Si noti come il periodo e la frequenza di un moto armonico sono indipendenti dall'ampiezza del moto.

			\paragraph{Classi di moti armonici}
			Fissato il valore della pulsazione si ottiene una classe di moti armonici caratterizzata dallo stesso periodo che differiscono tra loro per i diversi valori dell'ampiezza e della fase iniziale, ovvero per le condizioni iniziali.

		\subsubsection{Velocit\`a}
		La velocit\`a del punto che si muove con moto armonico si ottiene derivando $x(t)$.
		\begin{align*}
			\dot{x} &= v(t) = \dfrac{dx}{dt}\\
			  &=\omega A \cos(\omega t+\phi)\\
		\end{align*}
		La velocit\`a assume il valore massimo nel centro di oscillazione dove vale $\omega A$ e si annulla agli estremi dove si inverte il senso del moto.

		\subsubsection{Accelerazione}
		L'accelerazione del punto che si muove con moto armonico si ottiene derivando $v(t)$.
		\begin{align*}
			\ddot{x} &= a(t) = \dfrac{dv}{dt} = \dfrac{d^2x}{dt^2}\\
			       &= -\omega^2 A \sin(\omega t + \phi)\\
			       &= -\omega^2 x
		\end{align*}
		L'accelerazione si annulla nel centro di oscillazione e assume il valore in modulo massimo $\omega^2 A$ agli estremi, dove si inverte la velocit\`a.
		Si nota inoltre come sia proposizionale e d opposta allo spostamento dal centro di oscillazione.
		A parte il valore dell'ampiezza le tre funzioni mostrano lo stesso andamento temporale, si nota unicamente uno spostamento di una rispetto all'altra lungo l'asse dei tempi.
		Si nota pertanto come la velocit\`a sia sfasata di $\frac{\pi}{2}$  rispetto allo spostamento o si trova in quadratura di fase, mentre l'accelerazoine \`e sfasata di $\pi$ rispetto allo spostamento o si trova in opposizione di fase.

		\subsubsection{Condizioni iniziali}
		Le costanti $A$ e $\phi$ identificano le condizioni iniziali:
		$$ x(0) = x_0 = A \sin\phi$$
		$$v(0) = v_0 = \omega A \cos\phi$$
		Note le condizioni iniziali $x_0$ e $v_0$ si calcolano $A$ e $\phi$ come:
		$$\tan \phi = \dfrac{\omega x_0}{v_0}$$
		$$A^2 = x_0^2 + \dfrac{v_0^2+}{\omega^2}$$

		\subsubsection{Dipendenza della velocit\`a dalla posizione}
		\begin{align*}
			v(x) &= \int_{x_o}^x a(x)dx\\
			     &=-\omega^2\int_{x_0}^x xdx\\
			     &=\dfrac{1}{2}\omega^2(x_x^2-x^2)\\
			     &=\dfrac{1}{2}v^2 - \dfrac{1}{2}v_0^2
		\end{align*}
		Pertanto
		$$v^2 = v_0^2 + \omega^2(x_0^2(x_0^2-x^2)$$
		Con riferimento al centro dove $x_0 = 0$ e $v_0 = \omega A$
		$$v^2(x)=\omega^2(A^2 - x^2)$$
		Il segno di $v$ dipende dal verso di passaggio.
		Si nota pertanto come l'accelerazione \`e proporzionale allo spostamento con segno negativo $a =-\omega^2 x$.

		\subsubsection{Condizione sufficiente per un moto armonico semplice}
		Se si trova che in un moto l'accelerazione \`e proporzionale allo spostamento con costante di proporzionalit\`a negativa si dimostra che quel moto \`e armonico semplice.
		La condizione necessaria e sufficiente affinch\`e un moto sia armonico \`e:
		$$\dfrac{d^2x}{dt^2}+\omega^2x=0$$
		O equazione differenziale del moto armonico.
		Le funzioni seno e coseno e le loro combinazioni lineari sono tutte e sole le funzioni che soddisfano la condizione nel campo reale.

			\paragraph{Moto con funzione coseno}
			Queste considerazioni portano a considerare una legge del moto che utilizzi la funzione coseno.
			Si noti come le due funzioni differiscono per un termine di sfasamento $\frac{\pi}{2}$.
			Ovvero $x=A\sin(\omega t+\phi)$ e $x = A\cos(\omega t+\phi)$ rappresentano lo stesso moto, solo che il primo \`e visto a partire dall'istante $t_0$, mentre il secondo dall'istante $t_0 + \dfrac{T}{4}$.

		\subsubsection{Oscillazione}
		Se in un diverso fenomeno fisico si trova una grandezza $f$ che obbedisce a
		$$\dfrac{d^2f}{dz^2}+k^2f=0$$
		La soluzione \`e sempre:
		$$f(z)=A\sin(kz+\phi)$$
		Ovvero $f$ descrive un'oscillazione rispetto a $z$ il cui periodo dipende da $k$.

	\subsection{Moto rettilineo smorzato esponenzialmente}
	Si consideri ora un altro moto vario in cui l'accelerazione soddisfa la condizione $a = -kv$, con $k$ costante positiva.
	L'accelerazione \`e sempre contraria alla velocit\`a che deve necessariamente diminuire e varia ocn la stessa legge con cui varia la velocit\`a, ovvero:
	$$\dfrac{dv}{dt} = -kv$$
	Integrando con il metodo della separazione delle variabili:
	\begin{align*}
		&\dfrac{dv}{v} = -kdt\\
		\Rightarrow&\int_{v_0}^v\dfrac{dv}{v} = -k\int_0^t\\
		\Rightarrow &\log\dfrac{v}{v_0} = -kt
	\end{align*}
	Dove $v_0$ \`e la velocit\`a in $t=0$ e $v_0\neq 0$.
	Passando alle esponenziali:
	$$v(t) = v_0e^{-kt}$$
	La velocit\`a decresce esponenzialmente nel tempo e il punto alla fine si ferma.

		\subsubsection{Cambio della velocit\`a con la posizione}
